\begin{abstract}
Este trabalho consiste na análise de uma organização, onde vai ser efetuado sua descrição e um diagnostico da \textcolor{blue}{Cultura Organizacional} aplicando o modelo de \textcolor{blue}{Ogbonna \& Harris} e uma reflexão do seu impacto na organização e nos colaboradores.
\emptyline
A Cultura Organizacional é fundamental para a organização poder evoluir e atingir seus objetivos com sucesso. O estudo da cultura presente na organização e a forma de a moldar para melhor servir a sociedade e mercado será abordado neste relatório.
\emptyline
As organizações que tem maior sucesso são as que tem uma cultura inovadora ou competitiva com uma liderança orientada as pessoas e ou participativa, as empresas devem ter uma identidade que lhe é própria e estabelecer objetivos claros, concisos, calendarizados e atingíveis, seus colaboradores devem estar alinhados com os objetivos e ter a oportunidade de crescer, autoconfiança e um ambiente que promove o sucesso. Empresas devem estar sempre focadas aos exterior, para o seu mercado exigente.
\emptyspace{13cm}
\textbf{Palavras Chave:} Comportamento Organizacional, Cultura Organizacional, Comunicação, Motivação, Tomada de Decisão, Liderança, Planeamento, Organização, Controlo.
\end{abstract}
