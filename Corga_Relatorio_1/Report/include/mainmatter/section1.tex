\section{Introdução}
\qquad Este trabalho consiste no estudo da cultura organizacional, vai se distinguir entre quatro tipos de culturas e determinar por inquérito qual é possivelmente mais predominante na empresa S.Roque.
\emptyline
Vai ser apresentado uma revisão da matéria, uma apresentação da empresa S.Roque e utilizado o modelo Ogbonna \& Harris para seu estudo, no fim conclusões.
\emptyline
A S.Roque é uma organização privada com fins lucrativos, que tem de oferecer valor aos seus (potenciais) clientes, aos detentores do seu capital, aos trabalhadores e aos fornecedores, existe um forte estimulo do mercado (e da concorrência) estabelece objetivos bem definidos e socialmente apetecidos para poder sobreviver e prosperar.\cite{book_10}
\emptyline
A S.Roque tem uma hierarquia achatada com vários departamentos de suporte e produtivo, um leque de produtos (maquinas) da área têxtil dedicado a estampagem que são la desenvolvidos e montados, isto constituí a sua atividade principal, presta serviços aos seus produtos, também serve de apoio a outras empresas que recorrem aos seus serviços sectoriais da produção (outsourcing) para executar tarefas por encomenda, como a Roque Lazer e Serralharia (ex: Corte a laser (chapas), quinadeira, soldadura, etc)
\emptyline
Uma Organização Internacional com uma equipa de Marketing e de Vendas eficaz.
\emptyline
O sector produtivo tem uma implementação por produto com uma divisão de trabalho na qual é eficiente, e um bom controlo da qualidade.
\emptyline
Hoje em dia nenhuma organização vive em ambiente estável, e tem que estar sempre a analisar o mercado externo e estar sempre em constante adaptação ao mercado exigente, como a S.Roque tem vindo a fazer recorrendo ao mercado externo ao país e criando novos produtos com procura competitiva. É uma empresa inovadora procura sempre aperfeiçoar seus produtos e utilizando as ferramentas mais recentes tais como o \textbf{Solid Works}.
\emptyline
O modelo Ogbonna \& Harris é complexo tem muitos parâmetros, foi aplicado a 322 companhias na Inglaterra, sua cultura é diferente da de Portugal, e não se sabe se o resultado seria igual.\\
Enquanto o processo foi de retirar dados de varias companhias de forma a obter um modelo que supostamente poderia representar as organizações no país, tirando conclusões interessantes e indicadores quais tem mais sucesso, neste relatório só se vai aplicar um inquérito a uma empresa para determinar sua cultura organizacional.
%%%%%%%%%%%%%%%%%%%%%%%%%%%%%%%%%%%%%%%%%%%%%%%%%%%%%%%%%%%%%%%%%%%%%%%%%%%%%%%%%