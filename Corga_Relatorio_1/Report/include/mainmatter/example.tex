\gls{gloss}
\ac{twi}
\todo[inline]{Most up to date model 14052025}
\lstinputlisting[language=C, firstline=1, lastline=10, style=CLanguageStyle]{./input/mainmatter/language/lcd_1/lcd.c}
%%%%%%%%%%%%%%%%%%%%%%%%%%%%%%%%%%%%%%%%%%%%%%%%%%%%%%%%%%%%%%%%%%%%%%%%%%%%%%%%%
\emptyspace{3cm}
%%%%%%%%%%%%%%%%%%%%%%%%%%%%%%%%%%%%%%%%%%%%%%%%%%%%%%%%%%%%%%%%%%%%%%%%%%%%%%%%%
%\begin{comment}
Usando o método Socrático podemos perguntar qual a melhor ou melhores culturas organizacional que se enquadra na nacional nos vários tipos de organizações existentes.\\
%%%%%%%%%%%%%%%%%%%%%%%%%%%%
Para criar uma cultura na organização deve-se definir uma missão, valores, visão e os objetivos.\\
%%%%%%%%%%%%%%%%%%%%%%%%%%%%
Durante muito tempo tem havido estudos para descobrir a formula mágica que leva as organizações a ter sucesso, existe muitas abordagens com diferentes perspectivas, algumas convergem e outras divergem, e assim foram criados novos conceitos de forma a encapsular estilos e tipos de forma a poder se identificar quais são os mais propícios a ter uma maior elevada taxa de sucesso, conceitos tais como comportamento organizacional, cultura organizacional, performance organizacional, etc.\\
%%%%%%%%%%%%%%%%%%%%%%%%%%%%%
GENDER:
Nearly half of the U.S. workforce is now made up of women, and women are a growing percentage of the workforce in most countries throughout the world. Organizations need to ensure that hiring and employment policies create equal access and opportunities to individuals, regardless of gender.
RACE:
The percentage of Hispanics, blacks, and Asians in the U.S. workforce continues to increase. Organizations need to ensure that policies provide equal access and opportunities, regardless of race.
NATIONAL ORIGIN:
A growing percentage of U.S. workers are immigrants or come from homes where English is not the primary language spoken. Because employers in the United States have the right to demand that English be spoken at the workplace during job-related activities, communication problems can occur when employees’ English language skills are weak.
AGE:
The U.S. workforce is aging, and recent polls indicate that an increasing percentage of employees expect to work past the traditional retirement age of 65. Organizations cannot discriminate on the basis of age and need to make accommodations for the needs of older workers.
DISABILITY:
Organizations need to ensure that jobs and workplaces are accessible to the mentally, physically, and health challenged.
DOMESTIC PARTNERS
An increasing number of gay and lesbian employees, as well as employees with live-in partners of the opposite sex, are demanding the same rights and benefits for their partners that organizations have provided for traditional married couples.
RELIGION:
Organizations need to be sensitive to the customs, rituals, and holidays, as well as the appearance and attire, of individuals of non-Christian faiths such as Judaism, Islam, Hinduism, Buddhism, and Sikhism, and ensure that these individuals suffer no adverse impact as a result of their appearance or practices.
Tipos de descriminação:
sexual, intimidação, insultos e coerção, exclusão, incivilidade.
A satisfação melhora o desempenho.\\
%%%%%%%%%%%%%%%%%%%%%%%%%
Nas empresas publicas, o estimulo é sentido em menor grau ou não existente, consoante o tipo e características concretas da organização.\\
%%%%%%%%%%%%%%%%%%%%%%%%%
Nas organizações sem fim lucrativos e/ou dependentes do estado, existe outro tipo de estimulo os interesses dos governantes, quando o estado é parceiro ou responsável pela organização, os interesses de grupos de pressão da sociedade, etc.\\
%%%%%%%%%%%%%%%%%%%%%%%%%
Nas organizações sem fins lucrativos os objetivos estão em permanente discusão e/ou a ser alterados, resultando uma maior indefinição sobre as atividades a desenvolver por responsáveis e por colaboradores.\\
%%%%%%%%%%%%%%%%%%%%%%%%%
Teoria de Geer Hofteed.\\
%%%%%%%%%%%%%%%%%%%%%%%%%%
Best practices can come from national, say the American National Standards Institute (ANSI) or the Canadian Standards Association (CSA), or international, say ISO or Institute of Electrical and Electronics Engineers (IEEE), standards organizations, professional associa-tions, or consulting firms.\\
%%%%%%%%%%%%%%%%%%%%%%%%%
Eliminar desperdício e resolução de problemas.\\
%%%%%%%%%%%%%%%%%%%%%%%%%
Confiança na liderança e operadores.\\
%%%%%%%%%%%%%%%%%%%%%%%%%
Leaders have long been viewed as a primary influence on the creation of organizational culture (e.g. Bennis and Nanus, 1985; Schein, 1983). According to Schein (1985), the “only thing of real importance that leaders do is to create and manage culture”\\
%%%%%%%%%%%%%%%%%%%%%%%%%
Constante mudança de adaptação ao meio ambiente.\\
%%%%%%%%%%%%%%%%%%%%%%%%%
acknowledge that even in US and European companies, success rates are
not spectacular regarding efforts to change vision, values, and culture or business systems
and processes (Beer and Nohria, 2000; Beer et al., 1990; Carr et al., 1996).\\
%%%%%%%%%%%%%%%%%%%%%%%%%
Característica de bons Objetivos\\
- Claros\\
- Concisos\\
- Calendarizados\\
- Atingíveis\\
%%%%%%%%%%%%%%%%%%%%%%%%%
Tipos de organizações\\
- Organização privadas com fins lucrativos\\
- Organização privadas sem fins lucrativos\\
- Organização publicas com fins lucrativos\\
- Organização publicas sem fins lucrativos\\
%%%%%%%%%%%%%%%%%%%%%%%%%
tipos de hierarquias\\
tipos de departamentalizações\\
organização por processo\\
%%%%%%%%%%%%%%%%%%%%%%%%%
A divisão do trabalho, permitiu a redução do tempo de aprendizagem, isto é, cada um tem as suas funções, aumentando a produtividade. Cada um executa uma parte das tarefas necessárias a fabricação.\\
%%%%%%%%%%%%%%%%%%%%%%%%%
Gestão:
\emptyline
\begin{minipage}{20cm}
\begin{minipage}{5cm}
Instrumentos
\begin{enumerate}
\item Planear
\item Organizar
\item Controlar\\ \\
\end{enumerate}
\end{minipage}
\begin{minipage}{5cm}
Funções
\begin{enumerate}
\item Liderança
\item Comunicação
\item Motivação
\item Tomada de decisão
\end{enumerate}
\end{minipage}
\end{minipage}
%%%%%%%%%%%%%%%%%%%%%%%%%
Cadeia de valor\\
-Atividades principais\\
-Atividades de suporte\\
%%%%%%%%%%%%%%%%%%%%%%%%%
Cadeia de valor da organização é a sequencia de atividades e fluxos de informação que uma organização e os seus fornecedores devem desenvolver para desenhar, produzir, oferecer, entregar e suportar os seus produtos, estas são as atividades principais.\\
%%%%%%%%%%%%%%%%%%%%%%%%%
As atividades de suporte são as que apoiam um bom desempenho na realização das actividades principais.\\
- atividade administrativa e financeira\\
- atividade da gestão do pessoal\\
- atividade jurídica\\
- planeamento, controlo e gestão\\
- gestão de sistemas e tecnologia\\
%%%%%%%%%%%%%%%%%%%%%%%%%%
A atividade de suporte não contribuem diretamente para a criação do valor.\\
%%%%%%%%%%%%%%%%%%%%%%%%%%
Atividades de suporte e principal.\\
funções da Gestão sã Instrumental, Comportamental e Estrutural.\\
%%%%%%%%%%%%%%%%%%%%%%%%%%
Cumprir os objetivos é ser eficaz.\\
%%%%%%%%%%%%%%%%%%%%%%%%%%%
Para gerir a produção (planear, organizar, dirigir e controlar), há que recolher um elevado volume de informação de controlo, sendo frequentemente necessário refazer o planeamento.\\
%%%%%%%%%%%%%%%%%%%%%%%%%%%
- Implementação por projeto\\
- Implementação por processo\\
- Implementação por células\\
- Implementação por cadeia ou em linha\\
- Implementação por produto\\
%%%%%%%%%%%%%%%%%%%%%%%%%%
Na implementação por célula de fabrico procura agrupar os produtos segundo a semelhança das suas rotinas operatórias.\\
Na implementação por processo, é possível cada serie (ou lote) ser processado integralmente num dado centro, antes de avançar para o centro onde irá sofrer a operação de transformação seguinte.\\
A análise ABC pode ser utilizada para averiguar quais as principais encomendas responsáveis pela sobrecarga de um dado centro de trabalho.\\
%%%%%%%%%%%%%%%%%%%%%%%%%%
Organizar é estipular quem faz o quê, atribui-se os recursos necessários para o fazer, criar um sistema de informação para verificar execução.\\
%%%%%%%%%%%%%%%%%%%%%%%%%%%
O que lhes chama a atenção e medem; suas reações a incidentes criticos, alocação de meios, papeis assumidos, e partilha de informação; recompensas e delegação de poder; recrutamento, seleção e promoção. As lideranças chave tem como responsabilidade de modificar a cultura de forma a estar atualizada com as mudanças exigidas.\\
%%%%%%%%%%%%%%%%%%%%%%%%%%%%
Aqui distingue-se dois tipos de lideres os transacionais e os de transformação. “culture affects leadership  as much as leadership affects culture”
%%%%%%%%%%%%%%%%%%%%%%%%%%%%
Planear é estabelecer os objetivos a atingir e o percurso de ações.\\
%%%%%%%%%%%%%%%%%%%%%%%%%%%%
A formulação, avaliação e seleção de estratégias e o desenvolvimento dos planos mais detalhados para as pôr em prática são feitos após a definição da missão e da análise do meio ambiente da organização.\\
%%%%%%%%%%%%%%%%%%%%%%%%%%%%
Ferramentas para avaliar o cumprimento dos objetivos\\
- benchmarking\\
- scorecard management\\
- Banco de Portugal\\
%%%%%%%%%%%%%%%%%%%%%%%%%%%%
Método de demonstrar o desempenho de uma organizações através da eficacia, eficiência e seu rendimento.\\
%%%%%%%%%%%%%%%%%%%%%%%%%%%%%
Eficacia avalia em que medida os objetivos estão alinhados com a necessidades sociais que ela se propõe a satisfazer, ou seja, em que medida os seus objetivos são a tal adequados.\\
%%%%%%%%%%%%%%%%%%%%%%%%%%%%%
Eficiência avalia a economia de recursos utilizados para realizar os seus objetivos, requer uma boa estruturação dos processos seguidos nas atividades, o que leva tempo e custa dinheiro.\\
%%%%%%%%%%%%%%%%%%%%%%%%%%%%%
Missão - SWOT Meio Ambiente (transacional e contextual(PEST)) - Objetivos - Implementação.\\
%%%%%%%%%%%%%%%%%%%%%%%%%%%%%
O sucesso das empresas está correlacionada positivamente com o seu planeamento.\\
%%%%%%%%%%%%%%%%%%%%%%%%%%%%%
Na análise do meio ambiente transacional, analisa-se o comportamento previsional das entidades com quem a organização interage.\\
%%%%%%%%%%%%%%%%%%%%%%%%%%%%%
Controlo.\\
O controlo pode ser encarado como um processo de aprendizagem.\\
O controlo deve servir, acima de tudo, para ajudar a garantir que os objetivos estabelecidos são atingidos.\\
Se a informação recolhida e os resultados apurados no processo de controlo não conduzem a ações de correção quando necessário, este será não só inútil, mas até prejudicial.\\
O recurso aos sistemas de informação permite, em geral, simplificar os procedimentos de controlo.\\
%%%%%%%%%%%%%%%%%%%%%%%%%%%%%%%
Controlar é os procedimentos de verificar sua execução, estar atento a imprevistos e pronto a correções recorrendo a re-organização e/ou novo planeamento, também pode-se optar por não fazer nada.\\
%%%%%%%%%%%%%%%%%%%%%%%%%%%%%%%%
Estilos de liderança:
\begin{itemize}
\item participative leadership\\
Is gauged by the extent to which leaders allow subordinates to influence decisions by requesting input and contribution.
\item supportive leadership\\
Focuses on the degree to which the behaviour of a leader can be viewed as sympathetic, amicable, and considerate of subordinate needs.
\item Instrumetal leadership\\
This measure of leadership style is akin to directive or transactional leadership and is designed to measure the extent to which leaders specify expectations, establish procedures, and allocate tasks.
\end{itemize}
Motivação teorias Moslow, Hersberg, Victor Vroom.\\
Se as condições e recompensas oferecidas aos funcionários não lhes permitirem satisfazerem algumas das suas necessidades, mais facilmente abandonam a equipa ou organização a que pertencem.\\
Motivação é o conjunto de fatores que provocam, canalizam e sustentam o comportamento das pessoas.\\
Um gestor interessado em atingir um bom desempenho estabelece objetivos atingíveis e bem defendidos.\\
Auto-confiança no desenvolvimento do trabalho pode diminuir a sua motivação.\\
Motivar é criar condições necessárias para que as pessoas se empenham na prossecução dos objetivos da organização.\\
O impacto da comunicação no desempenho da organização é muito elevado.\\
Os Gestores tem de ser coerentes e alinhados com o que transmitem de forma a criar uma estrutura de confiança.\\
Um líder de uma organização é aquele que detém capacidades de influenciar os colaboradores.\\
gestão das atividades é o exercício do poder de um gestor.\\
poder de premiar e punir é suficiente para gerir as atividades do dia a dia.\\
poder informacional.\\
Um sistema de avaliação do desempenho de uma empresa é uma valia porque é uma boa oportunidade para analisar o grau de cumprimento dos objetivos acordados.\\
A medida que as organizações se achatam, os gestores têm de aprender a permitir que os seus colaboradores tomem decisões e tenham informação sobre questões mais sensíveis.\\
Para um engenheiro é muito útil conhecer os aspectos essenciais da legislação laboral.\\
A gestão das pessoas é cada vez mais importante porque são as pessoas que têm o conhecimento e só as pessoas o podem partilhar e aplicar.\\
Numa organização, apoiar e compensar as pessoas é fundamental, mas também é indispensável falar com elas sobre os erros que cometem no sentido de serem corrigidas e evitadas no futuro.\\
Uma organização tem maior probabilidade de ter sucesso se gerir as pessoas de modo a que estas ao contribuírem para o sucesso da organização tenham também sucesso elas próprias.\\
A qualidade é a totalidade das características de um produto ou serviço, que determinam a sua aptidão para satisfazer determinadas necessidades.\\
Garantia da qualidade tem como objetivo primeiro, o controlo do processo, ou seja, a minimização ou mesmo eliminação dos erros na produção.\\
A garantia da qualidade concentra-se no controlo do processo produtivo e controlo do produto.\\
Os custos relacionados com a insatisfação dos clientes são considerados custos de não qualidade.\\
O diagrama de Pareto permite identificar rapidamente as causas vitais e as triviais de um dado problema.\\
As cartas de controlo destinam-se a detetar as variações resultantes da alteração, frequentemente de natureza aleatória e acidental, de algum dos parâmetros de processo de fabrico (ditas causas especiais).\\
As sete ferramentas clássicas da qualidade\\
- Fluxograma\\
- Registo e análise de dados\\
- Diagrama de causa - efeito (espinha de peixe 4M)\\
- Diagrama de Pareto\\
- Histogramas\\
- Diagramas de dispersão\\
- Cartas de Controlo\\
tipos de lideres\\
- Autocrático\\
- Participativo \\
- Democrático \\
- Deixa andar\\
estilo de líder\\
- Orientado as pessoas\\
- Orientado as tarefas\\
%%%%%%%%%%%%%%%%%%%%%%%%%%%
The relationship between culture and leadership appears to be reciprocal—top leaders create and maintain an organizational culture, which in turn influences the values, attitudes, and behaviors of middle and entry-level leaders. Although leader–culture fit has not been specifically studied in the published literature, we believe that there is value in examining the match between a leader’s behaviors and the culture in which they work. While research hasn’t examined this at the level we discuss, current research does suggest that fit is important at the national level and at the leader–follower level. Expansion of this research will help determine what aspects of leader culture fit are determinants of leader and organizational effectiveness. Although there are a variety of approaches that researchers can take to examining leader–culture fit, we offer the following recommendations. First, although studies of perceived fit are of limited value, the ease of collecting this data should motivate researchers to start thinking about adding questions concerning leader–culture fit. Given the lack of published findings, this research can begin shaping our knowledge about this phenomenon. Second, while studies of subjective fit will be more important, researchers should utilize 360-degree measurement systems in order to also obtain the most objective fit indices possible. This practice will likely tell us more about the impact of fit than just examining leaders’ self-reports. Third, it is important to measure culture at the aggregate level in order to ensure that the actual values of the organization are being captured, not just the leader’s values. Although these recommendations may be difficult to achieve in practice, they offer the best hope of leveraging leader–culture fit for the future.\\
%%%%%%%%%%%%%%%%%%%%%%%%%%%%%%%%%
A atividade comportamental consiste em lidar com pessoas, comunicar instruções e receber feedback, propiciar a comunicação entre os terceiros, motivar, tomar decisões e criar condições para que os colaboradores também o possam fazer, assegurar a liderança para cumprir a execução.\\
%%%%%%%%%%%%%%%%%%%%%%%%%%%%%%%%%%
More specifically, it suggests that culture provides the normative bounds for transactional leaders to be effective and that transformational leaders influence culture through strategic decisions and vision, by celebrating success, and by identifying and rewarding employees. \\
%%%%%%%%%%%%%%%%%%%%%%%%%%%%%%%%%%
In sum, leadership and organizational culture are related, and further, the dynamics between these constructs impact organizational effectiveness. However, the theoretical work in this area largely outweighs the empirical, and we believe there is utility in adopting a “fit” perspective for further research in this area.\\
%%%%%%%%%%%%%%%%%%%%%%%%%%%%%%%%%%
In their meta-analysis, Kristof-Brown et al. (2005) identified five types of fit research that captured the majority of published studies: person–vocation fit, person–job fit, person–organization fit, person–group fit, and person–supervisor fit. Attempting to integrate all of these perceptions of fit, Jansen \& Kristof-Brown (2006) proposed a multidimensional theory of person–environment (PE) fit.\\
%%%%%%%%%%%%%%%%%%%%%%%%%%%%%%%%%%%
O alinhamento entre o líder e a cultura da organização influencia a sua eficacia.\\
%%%%%%%%%%%%%%%%%%%%%%%%%%%%%%%%%%%
A vida ensina e se não aprendemos ela insiste e persiste até morrermos.\\
A liberdade é medida pela quantidade de ética e moralidade presente na sociedade.\\
Respect is always earned never a given.\\
%\end{comment}
%%%%%%%%%%%%%%%%%%%%%%%%%%%%%%%%%%%%%%%%%%%%%%%%%%%%%%%%%%%%%%%%%%%%%
%%%%%%%%%%%%%%%%%%%%%%%%%%%%%%%%%%%%%%%%%%%%%%%%%%%%%%%%%%%%%%%%%%%%%
